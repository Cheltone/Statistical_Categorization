% Options for packages loaded elsewhere
\PassOptionsToPackage{unicode}{hyperref}
\PassOptionsToPackage{hyphens}{url}
%
\documentclass[
]{article}
\usepackage{lmodern}
\usepackage{amssymb,amsmath}
\usepackage{ifxetex,ifluatex}
\ifnum 0\ifxetex 1\fi\ifluatex 1\fi=0 % if pdftex
  \usepackage[T1]{fontenc}
  \usepackage[utf8]{inputenc}
  \usepackage{textcomp} % provide euro and other symbols
\else % if luatex or xetex
  \usepackage{unicode-math}
  \defaultfontfeatures{Scale=MatchLowercase}
  \defaultfontfeatures[\rmfamily]{Ligatures=TeX,Scale=1}
\fi
% Use upquote if available, for straight quotes in verbatim environments
\IfFileExists{upquote.sty}{\usepackage{upquote}}{}
\IfFileExists{microtype.sty}{% use microtype if available
  \usepackage[]{microtype}
  \UseMicrotypeSet[protrusion]{basicmath} % disable protrusion for tt fonts
}{}
\makeatletter
\@ifundefined{KOMAClassName}{% if non-KOMA class
  \IfFileExists{parskip.sty}{%
    \usepackage{parskip}
  }{% else
    \setlength{\parindent}{0pt}
    \setlength{\parskip}{6pt plus 2pt minus 1pt}}
}{% if KOMA class
  \KOMAoptions{parskip=half}}
\makeatother
\usepackage{xcolor}
\IfFileExists{xurl.sty}{\usepackage{xurl}}{} % add URL line breaks if available
\IfFileExists{bookmark.sty}{\usepackage{bookmark}}{\usepackage{hyperref}}
\hypersetup{
  pdftitle={HW3 Chad Melton and Rebecca Brink},
  hidelinks,
  pdfcreator={LaTeX via pandoc}}
\urlstyle{same} % disable monospaced font for URLs
\usepackage[margin=1in]{geometry}
\usepackage{color}
\usepackage{fancyvrb}
\newcommand{\VerbBar}{|}
\newcommand{\VERB}{\Verb[commandchars=\\\{\}]}
\DefineVerbatimEnvironment{Highlighting}{Verbatim}{commandchars=\\\{\}}
% Add ',fontsize=\small' for more characters per line
\usepackage{framed}
\definecolor{shadecolor}{RGB}{248,248,248}
\newenvironment{Shaded}{\begin{snugshade}}{\end{snugshade}}
\newcommand{\AlertTok}[1]{\textcolor[rgb]{0.94,0.16,0.16}{#1}}
\newcommand{\AnnotationTok}[1]{\textcolor[rgb]{0.56,0.35,0.01}{\textbf{\textit{#1}}}}
\newcommand{\AttributeTok}[1]{\textcolor[rgb]{0.77,0.63,0.00}{#1}}
\newcommand{\BaseNTok}[1]{\textcolor[rgb]{0.00,0.00,0.81}{#1}}
\newcommand{\BuiltInTok}[1]{#1}
\newcommand{\CharTok}[1]{\textcolor[rgb]{0.31,0.60,0.02}{#1}}
\newcommand{\CommentTok}[1]{\textcolor[rgb]{0.56,0.35,0.01}{\textit{#1}}}
\newcommand{\CommentVarTok}[1]{\textcolor[rgb]{0.56,0.35,0.01}{\textbf{\textit{#1}}}}
\newcommand{\ConstantTok}[1]{\textcolor[rgb]{0.00,0.00,0.00}{#1}}
\newcommand{\ControlFlowTok}[1]{\textcolor[rgb]{0.13,0.29,0.53}{\textbf{#1}}}
\newcommand{\DataTypeTok}[1]{\textcolor[rgb]{0.13,0.29,0.53}{#1}}
\newcommand{\DecValTok}[1]{\textcolor[rgb]{0.00,0.00,0.81}{#1}}
\newcommand{\DocumentationTok}[1]{\textcolor[rgb]{0.56,0.35,0.01}{\textbf{\textit{#1}}}}
\newcommand{\ErrorTok}[1]{\textcolor[rgb]{0.64,0.00,0.00}{\textbf{#1}}}
\newcommand{\ExtensionTok}[1]{#1}
\newcommand{\FloatTok}[1]{\textcolor[rgb]{0.00,0.00,0.81}{#1}}
\newcommand{\FunctionTok}[1]{\textcolor[rgb]{0.00,0.00,0.00}{#1}}
\newcommand{\ImportTok}[1]{#1}
\newcommand{\InformationTok}[1]{\textcolor[rgb]{0.56,0.35,0.01}{\textbf{\textit{#1}}}}
\newcommand{\KeywordTok}[1]{\textcolor[rgb]{0.13,0.29,0.53}{\textbf{#1}}}
\newcommand{\NormalTok}[1]{#1}
\newcommand{\OperatorTok}[1]{\textcolor[rgb]{0.81,0.36,0.00}{\textbf{#1}}}
\newcommand{\OtherTok}[1]{\textcolor[rgb]{0.56,0.35,0.01}{#1}}
\newcommand{\PreprocessorTok}[1]{\textcolor[rgb]{0.56,0.35,0.01}{\textit{#1}}}
\newcommand{\RegionMarkerTok}[1]{#1}
\newcommand{\SpecialCharTok}[1]{\textcolor[rgb]{0.00,0.00,0.00}{#1}}
\newcommand{\SpecialStringTok}[1]{\textcolor[rgb]{0.31,0.60,0.02}{#1}}
\newcommand{\StringTok}[1]{\textcolor[rgb]{0.31,0.60,0.02}{#1}}
\newcommand{\VariableTok}[1]{\textcolor[rgb]{0.00,0.00,0.00}{#1}}
\newcommand{\VerbatimStringTok}[1]{\textcolor[rgb]{0.31,0.60,0.02}{#1}}
\newcommand{\WarningTok}[1]{\textcolor[rgb]{0.56,0.35,0.01}{\textbf{\textit{#1}}}}
\usepackage{graphicx,grffile}
\makeatletter
\def\maxwidth{\ifdim\Gin@nat@width>\linewidth\linewidth\else\Gin@nat@width\fi}
\def\maxheight{\ifdim\Gin@nat@height>\textheight\textheight\else\Gin@nat@height\fi}
\makeatother
% Scale images if necessary, so that they will not overflow the page
% margins by default, and it is still possible to overwrite the defaults
% using explicit options in \includegraphics[width, height, ...]{}
\setkeys{Gin}{width=\maxwidth,height=\maxheight,keepaspectratio}
% Set default figure placement to htbp
\makeatletter
\def\fps@figure{htbp}
\makeatother
\setlength{\emergencystretch}{3em} % prevent overfull lines
\providecommand{\tightlist}{%
  \setlength{\itemsep}{0pt}\setlength{\parskip}{0pt}}
\setcounter{secnumdepth}{-\maxdimen} % remove section numbering

\title{HW3 Chad Melton and Rebecca Brink}
\author{}
\date{\vspace{-2.5em}}

\begin{document}
\maketitle

\begin{Shaded}
\begin{Highlighting}[]
\KeywordTok{library}\NormalTok{(epiR)}
\end{Highlighting}
\end{Shaded}

\begin{verbatim}
## Loading required package: survival
\end{verbatim}

\begin{verbatim}
## Package epiR 1.0-15 is loaded
\end{verbatim}

\begin{verbatim}
## Type help(epi.about) for summary information
\end{verbatim}

\begin{verbatim}
## Type browseVignettes(package = 'epiR') to learn how to use epiR for applied epidemiological analyses
\end{verbatim}

\begin{verbatim}
## 
\end{verbatim}

\begin{Shaded}
\begin{Highlighting}[]
\KeywordTok{library}\NormalTok{(ggplot2)}
\KeywordTok{library}\NormalTok{(vcdExtra)}
\end{Highlighting}
\end{Shaded}

\begin{verbatim}
## Loading required package: vcd
\end{verbatim}

\begin{verbatim}
## Loading required package: grid
\end{verbatim}

\begin{verbatim}
## Loading required package: gnm
\end{verbatim}

\begin{Shaded}
\begin{Highlighting}[]
\KeywordTok{library}\NormalTok{(vcd)}
\KeywordTok{library}\NormalTok{(MASS)}
\KeywordTok{library}\NormalTok{(lmtest)}
\end{Highlighting}
\end{Shaded}

\begin{verbatim}
## Loading required package: zoo
\end{verbatim}

\begin{verbatim}
## 
## Attaching package: 'zoo'
\end{verbatim}

\begin{verbatim}
## The following objects are masked from 'package:base':
## 
##     as.Date, as.Date.numeric
\end{verbatim}

\hypertarget{section}{%
\subsection{3.2}\label{section}}

\begin{enumerate}
\def\labelenumi{\alph{enumi})}
\item
\end{enumerate}

piHat = -0.0003 + 0.0304x and piHat = P\%xi. Now, 0.0304 = 0.01P

\begin{Shaded}
\begin{Highlighting}[]
\NormalTok{P =}\StringTok{ }\FloatTok{0.0304}\OperatorTok{/}\FloatTok{0.01}
\NormalTok{P}
\end{Highlighting}
\end{Shaded}

\begin{verbatim}
## [1] 3.04
\end{verbatim}

\begin{enumerate}
\def\labelenumi{\alph{enumi})}
\setcounter{enumi}{1}
\item
\end{enumerate}

\begin{Shaded}
\begin{Highlighting}[]
\NormalTok{piHat =}\StringTok{ }\FloatTok{0.0079}
\NormalTok{xi =}\StringTok{ }\FloatTok{0.0774}
\NormalTok{Pi =}\StringTok{ }\NormalTok{xi}\OperatorTok{*}\FloatTok{0.0304-0.0003}
\NormalTok{Pi}
\end{Highlighting}
\end{Shaded}

\begin{verbatim}
## [1] 0.00205296
\end{verbatim}

\begin{Shaded}
\begin{Highlighting}[]
\NormalTok{rat =}\StringTok{ }\NormalTok{piHat}\OperatorTok{/}\NormalTok{Pi}
\NormalTok{rat}
\end{Highlighting}
\end{Shaded}

\begin{verbatim}
## [1] 3.848102
\end{verbatim}

This calculation suggests that the value of Pii (0.0079) is an outlier
in the data because the 3.8\textgreater\textgreater0.0079.

\hypertarget{section-1}{%
\subsection{3.3}\label{section-1}}

\begin{enumerate}
\def\labelenumi{\alph{enumi})}
\item
\end{enumerate}

From the model selected we know that the equation follows the form
\(\pi \hat =\) intercept +parameter coefficient * parameter. From the
table given, we can see that the prediction equation is:
\[ \hat \pi  =0.00255 +0.00109x  \] where x is the alcohol consumption.
The intercept of this equation means that the probability of a child
having sex organ malformation without the influence of alcohol is only
about 0.00255. This increases as a rate of 0.00109 which means that the
probability of sex organ malformation doubles with every increase of 2
in alcohol consumption.

\begin{enumerate}
\def\labelenumi{\alph{enumi})}
\setcounter{enumi}{1}
\item
\end{enumerate}

First step in computing the relative risk is to calculate the
probability of malformation at each alcohol level. Do this by plugging
in 0 and 7 for x.

For \(x=0\) we have \(\hat \pi_0 = 0.00255 +0.00109(0) =0.00255\) and

For \(x=7\) we have \(\hat \pi_7 = 0.00255 +0.00109(7)=0.01018\).

The relative risk is then
\(\frac{\hat\pi_7}{\hat \pi_0}=\frac{0.01018}{0.00255}=4.0\)

\hypertarget{section-2}{%
\subsection{3.4}\label{section-2}}

To set up this problem, we first have to find a linear model that
satisfies the conditions set up in problem 3.3. Our model may be
slightly different as teh text book used SAS and we are using R. We have
also opted to only use the percentage of arrests data so our numbers are
different. This first linear model is set up by the code below.

\begin{Shaded}
\begin{Highlighting}[]
\NormalTok{consumption=}\StringTok{ }\KeywordTok{c}\NormalTok{(}\DecValTok{0}\NormalTok{,}\FloatTok{0.5}\NormalTok{,}\FloatTok{1.5}\NormalTok{,}\FloatTok{4.0}\NormalTok{,}\FloatTok{7.0}\NormalTok{)}
\NormalTok{percentage=}\KeywordTok{c}\NormalTok{(}\FloatTok{0.0028}\NormalTok{,}\FloatTok{0.0026}\NormalTok{,}\FloatTok{0.0063}\NormalTok{,}\FloatTok{0.0079}\NormalTok{,}\FloatTok{0.0263}\NormalTok{)}
\NormalTok{data1=}\KeywordTok{data.frame}\NormalTok{(consumption,percentage)}
\NormalTok{data1.model=}\KeywordTok{lm}\NormalTok{(percentage}\OperatorTok{~}\NormalTok{consumption,}\DataTypeTok{data=}\NormalTok{data1)}
\KeywordTok{summary}\NormalTok{(data1.model)}
\end{Highlighting}
\end{Shaded}

\begin{verbatim}
## 
## Call:
## lm(formula = percentage ~ consumption, data = data1)
## 
## Residuals:
##          1          2          3          4          5 
##  1.884e-03  9.451e-05  6.162e-04 -5.730e-03  3.135e-03 
## 
## Coefficients:
##              Estimate Std. Error t value Pr(>|t|)  
## (Intercept) 0.0009163  0.0024944   0.367   0.7377  
## consumption 0.0031783  0.0006789   4.682   0.0184 *
## ---
## Signif. codes:  0 '***' 0.001 '**' 0.01 '*' 0.05 '.' 0.1 ' ' 1
## 
## Residual standard error: 0.003941 on 3 degrees of freedom
## Multiple R-squared:  0.8796, Adjusted R-squared:  0.8395 
## F-statistic: 21.92 on 1 and 3 DF,  p-value: 0.01841
\end{verbatim}

This model is assumed to be our baseline.

\begin{enumerate}
\def\labelenumi{\alph{enumi})}
\item
\end{enumerate}

To see if the result is sensitive to the final data point where the
percentage of the malformations is an order of magnitude higher due to
only one malformation, we re-fit the same model assuming no
malformations in this category. This is set up below.

\begin{Shaded}
\begin{Highlighting}[]
\NormalTok{consumption2=}\KeywordTok{c}\NormalTok{(}\DecValTok{0}\NormalTok{,}\FloatTok{0.5}\NormalTok{,}\FloatTok{1.5}\NormalTok{,}\FloatTok{4.0}\NormalTok{,}\FloatTok{7.0}\NormalTok{)}
\NormalTok{percentage2=}\KeywordTok{c}\NormalTok{(}\FloatTok{0.0028}\NormalTok{,}\FloatTok{0.0026}\NormalTok{,}\FloatTok{0.0063}\NormalTok{,}\FloatTok{0.0079}\NormalTok{,}\FloatTok{0.0000}\NormalTok{)}
\NormalTok{data2=}\KeywordTok{data.frame}\NormalTok{(consumption2,percentage)}
\NormalTok{data2.model=}\KeywordTok{lm}\NormalTok{(percentage2}\OperatorTok{~}\NormalTok{consumption2,}\DataTypeTok{data=}\NormalTok{data2)}
\KeywordTok{summary}\NormalTok{(data2.model)}
\end{Highlighting}
\end{Shaded}

\begin{verbatim}
## 
## Call:
## lm(formula = percentage2 ~ consumption2, data = data2)
## 
## Residuals:
##         1         2         3         4         5 
## -0.001784 -0.001857  0.002099  0.004338 -0.002796 
## 
## Coefficients:
##                Estimate Std. Error t value Pr(>|t|)
## (Intercept)   0.0045843  0.0022427   2.044    0.134
## consumption2 -0.0002555  0.0006104  -0.419    0.704
## 
## Residual standard error: 0.003543 on 3 degrees of freedom
## Multiple R-squared:  0.05518,    Adjusted R-squared:  -0.2598 
## F-statistic: 0.1752 on 1 and 3 DF,  p-value: 0.7037
\end{verbatim}

The intercept for the model is about 4 times higher, but more
importantly the effect of alcohol consumption in the model is negative.
This would imply that drinking more alcohol reduces the chance of
malfomation in newborns. I am no pediatrics researcher, but I am pretty
sure that shouldn't be the case. This means that the one malformation we
see at the largest alcohol consumption rate is an important data point
to the dataset.

\begin{enumerate}
\def\labelenumi{\alph{enumi})}
\setcounter{enumi}{1}
\item
\end{enumerate}

We again try to see the sensitivity of the model, but this time to the
values of scores. If we change the values of the scores, we expect the
intercept to remain fairly constant and the parameter coefficeint to
shift slightly. We set up this model in the code below.

\begin{Shaded}
\begin{Highlighting}[]
\NormalTok{consumption3=}\KeywordTok{c}\NormalTok{(}\DecValTok{0}\NormalTok{,}\DecValTok{1}\NormalTok{,}\DecValTok{2}\NormalTok{,}\DecValTok{3}\NormalTok{,}\DecValTok{4}\NormalTok{)}
\NormalTok{percentage=}\KeywordTok{c}\NormalTok{(}\FloatTok{0.0028}\NormalTok{,}\FloatTok{0.0026}\NormalTok{,}\FloatTok{0.0063}\NormalTok{,}\FloatTok{0.0079}\NormalTok{,}\FloatTok{0.0263}\NormalTok{)}
\NormalTok{data3=}\KeywordTok{data.frame}\NormalTok{(consumption3,percentage)}
\NormalTok{data3.model=}\KeywordTok{glm}\NormalTok{(percentage}\OperatorTok{~}\NormalTok{consumption3,}\DataTypeTok{data=}\NormalTok{data2,)}
\KeywordTok{summary}\NormalTok{(data3.model)}
\end{Highlighting}
\end{Shaded}

\begin{verbatim}
## 
## Call:
## glm(formula = percentage ~ consumption3, data = data2)
## 
## Deviance Residuals: 
##        1         2         3         4         5  
##  0.00408  -0.00135  -0.00288  -0.00651   0.00666  
## 
## Coefficients:
##               Estimate Std. Error t value Pr(>|t|)  
## (Intercept)  -0.001280   0.004764  -0.269   0.8056  
## consumption3  0.005230   0.001945   2.689   0.0745 .
## ---
## Signif. codes:  0 '***' 0.001 '**' 0.01 '*' 0.05 '.' 0.1 ' ' 1
## 
## (Dispersion parameter for gaussian family taken to be 3.7833e-05)
## 
##     Null deviance: 0.00038703  on 4  degrees of freedom
## Residual deviance: 0.00011350  on 3  degrees of freedom
## AIC: -33.276
## 
## Number of Fisher Scoring iterations: 2
\end{verbatim}

We see that our parameter coefficient shifted by an order of magnitude.
This seems to be a large jump for a max change of scores from 7 to 4,
however when we look at the intercept it makes sense that the
coefficeint changed that much. The intercept is negative. This means
that not drinking any alcohol should lower the risk of drinking past
zero and ensure no malformations in newborns. This does not make logical
sense. A model of this data set is dependent on accurate values of the
number of drinks consumed and the proportions of malformations at every
level regardless of how high.

\begin{enumerate}
\def\labelenumi{\alph{enumi})}
\setcounter{enumi}{2}
\tightlist
\item
  Below is the code to fit the required model.
\end{enumerate}

\begin{Shaded}
\begin{Highlighting}[]
\NormalTok{consumption=}\StringTok{ }\KeywordTok{c}\NormalTok{(}\DecValTok{0}\NormalTok{,}\FloatTok{0.5}\NormalTok{,}\FloatTok{1.5}\NormalTok{,}\FloatTok{4.0}\NormalTok{,}\FloatTok{7.0}\NormalTok{)}
\NormalTok{percentage=}\KeywordTok{c}\NormalTok{(}\FloatTok{0.0028}\NormalTok{,}\FloatTok{0.0026}\NormalTok{,}\FloatTok{0.0063}\NormalTok{,}\FloatTok{0.0079}\NormalTok{,}\FloatTok{0.0263}\NormalTok{)}
\NormalTok{data4=}\KeywordTok{data.frame}\NormalTok{(consumption,percentage)}
\NormalTok{data4.model=}\KeywordTok{glm}\NormalTok{(percentage}\OperatorTok{~}\NormalTok{consumption,}\DataTypeTok{data=}\NormalTok{data4, }\DataTypeTok{family =} \KeywordTok{binomial}\NormalTok{(}\DataTypeTok{link =} \StringTok{"probit"}\NormalTok{))}
\end{Highlighting}
\end{Shaded}

\begin{verbatim}
## Warning in eval(family$initialize): non-integer #successes in a binomial glm!
\end{verbatim}

\begin{Shaded}
\begin{Highlighting}[]
\KeywordTok{summary}\NormalTok{(data4.model)}
\end{Highlighting}
\end{Shaded}

\begin{verbatim}
## 
## Call:
## glm(formula = percentage ~ consumption, family = binomial(link = "probit"), 
##     data = data4)
## 
## Deviance Residuals: 
##         1          2          3          4          5  
##  0.002670  -0.010808   0.024748  -0.025364   0.007384  
## 
## Coefficients:
##             Estimate Std. Error z value Pr(>|z|)
## (Intercept)  -2.7868     3.4482  -0.808    0.419
## consumption   0.1185     0.6698   0.177    0.860
## 
## (Dispersion parameter for binomial family taken to be 1)
## 
##     Null deviance: 0.0354303  on 4  degrees of freedom
## Residual deviance: 0.0014343  on 3  degrees of freedom
## AIC: 4.0926
## 
## Number of Fisher Scoring iterations: 8
\end{verbatim}

This should follow the equation: Probit(PiHat = -2.7860 +0.0614x.

\hypertarget{section-3}{%
\subsection{3.13}\label{section-3}}

\begin{enumerate}
\def\labelenumi{\alph{enumi})}
\item
\end{enumerate}

\begin{Shaded}
\begin{Highlighting}[]
\NormalTok{crab =}\StringTok{ }\KeywordTok{read.csv}\NormalTok{(}\StringTok{"C:}\CharTok{\textbackslash{}\textbackslash{}}\StringTok{Users}\CharTok{\textbackslash{}\textbackslash{}}\StringTok{chadm}\CharTok{\textbackslash{}\textbackslash{}}\StringTok{OneDrive}\CharTok{\textbackslash{}\textbackslash{}}\StringTok{Documents}\CharTok{\textbackslash{}\textbackslash{}}\StringTok{StatsData}\CharTok{\textbackslash{}\textbackslash{}}\StringTok{crab.csv"}\NormalTok{,}\DataTypeTok{header=}\OtherTok{TRUE}\NormalTok{)}

\NormalTok{crab.model =}\StringTok{ }\KeywordTok{glm}\NormalTok{(satell}\OperatorTok{~}\NormalTok{weight,}\DataTypeTok{data=}\NormalTok{crab,}\DataTypeTok{family =} \KeywordTok{poisson}\NormalTok{(}\DataTypeTok{link =}\StringTok{"log"}\NormalTok{))}
\KeywordTok{summary}\NormalTok{(crab.model)}
\end{Highlighting}
\end{Shaded}

\begin{verbatim}
## 
## Call:
## glm(formula = satell ~ weight, family = poisson(link = "log"), 
##     data = crab)
## 
## Deviance Residuals: 
##     Min       1Q   Median       3Q      Max  
## -2.9307  -1.9981  -0.5627   0.9298   4.9992  
## 
## Coefficients:
##             Estimate Std. Error z value Pr(>|z|)    
## (Intercept) -0.42841    0.17893  -2.394   0.0167 *  
## weight       0.58930    0.06502   9.064   <2e-16 ***
## ---
## Signif. codes:  0 '***' 0.001 '**' 0.01 '*' 0.05 '.' 0.1 ' ' 1
## 
## (Dispersion parameter for poisson family taken to be 1)
## 
##     Null deviance: 632.79  on 172  degrees of freedom
## Residual deviance: 560.87  on 171  degrees of freedom
## AIC: 920.16
## 
## Number of Fisher Scoring iterations: 5
\end{verbatim}

The equation is: log(muHat) = -0.42841 + 0.58930x

\begin{enumerate}
\def\labelenumi{\alph{enumi})}
\setcounter{enumi}{1}
\item
\end{enumerate}

\begin{Shaded}
\begin{Highlighting}[]
\NormalTok{x=}\FloatTok{2.44}

\NormalTok{muHat =}\StringTok{ }\KeywordTok{exp}\NormalTok{(}\OperatorTok{-}\FloatTok{0.42841} \OperatorTok{+}\StringTok{ }\FloatTok{0.58930}\OperatorTok{*}\NormalTok{x)}
\end{Highlighting}
\end{Shaded}

The mean Y for female crabs of average weight 2.44 kg is 2.744179.

\begin{enumerate}
\def\labelenumi{\alph{enumi})}
\setcounter{enumi}{2}
\item
\end{enumerate}

\begin{Shaded}
\begin{Highlighting}[]
\KeywordTok{confint}\NormalTok{(crab.model)}
\end{Highlighting}
\end{Shaded}

\begin{verbatim}
## Waiting for profiling to be done...
\end{verbatim}

\begin{verbatim}
##                  2.5 %      97.5 %
## (Intercept) -0.7771762 -0.07591025
## weight       0.4597002  0.71449835
\end{verbatim}

\begin{Shaded}
\begin{Highlighting}[]
\NormalTok{ciLow =}\StringTok{ }\KeywordTok{exp}\NormalTok{(}\FloatTok{0.4597}\NormalTok{)}
\NormalTok{ciHi =}\StringTok{ }\KeywordTok{exp}\NormalTok{(}\FloatTok{0.7167}\NormalTok{)}
\end{Highlighting}
\end{Shaded}

The CI for muHat is (1.583599, 2.047665).

\begin{enumerate}
\def\labelenumi{\alph{enumi})}
\setcounter{enumi}{3}
\item
\end{enumerate}

\begin{Shaded}
\begin{Highlighting}[]
\KeywordTok{waldtest}\NormalTok{(crab.model)}
\end{Highlighting}
\end{Shaded}

\begin{verbatim}
## Wald test
## 
## Model 1: satell ~ weight
## Model 2: satell ~ 1
##   Res.Df Df      F    Pr(>F)    
## 1    171                        
## 2    172 -1 82.155 2.856e-16 ***
## ---
## Signif. codes:  0 '***' 0.001 '**' 0.01 '*' 0.05 '.' 0.1 ' ' 1
\end{verbatim}

The Wald test provides a value of 82.155 and one degree of freedom.
Here, the P-Value is also \textasciitilde0 which suggest that there is
very strong evidence to reject the null hypothesis and shows that weight
is significant.

\begin{enumerate}
\def\labelenumi{\alph{enumi})}
\setcounter{enumi}{4}
\item
\end{enumerate}

\begin{Shaded}
\begin{Highlighting}[]
\KeywordTok{lrtest}\NormalTok{ (crab.model)}
\end{Highlighting}
\end{Shaded}

\begin{verbatim}
## Likelihood ratio test
## 
## Model 1: satell ~ weight
## Model 2: satell ~ 1
##   #Df  LogLik Df  Chisq Pr(>Chisq)    
## 1   2 -458.08                         
## 2   1 -494.04 -1 71.925  < 2.2e-16 ***
## ---
## Signif. codes:  0 '***' 0.001 '**' 0.01 '*' 0.05 '.' 0.1 ' ' 1
\end{verbatim}

The likelihood ratio tests provides a value of 71.295. Again, these
results suggest strong evidence to reject the null hypothesis.

\hypertarget{section-4}{%
\subsection{3.14}\label{section-4}}

\begin{enumerate}
\def\labelenumi{\alph{enumi})}
\item
\end{enumerate}

\begin{Shaded}
\begin{Highlighting}[]
\NormalTok{crab.model2 =}\StringTok{ }\KeywordTok{glm.nb}\NormalTok{(satell}\OperatorTok{~}\NormalTok{weight,}\DataTypeTok{data=}\NormalTok{crab)}
\KeywordTok{summary}\NormalTok{(crab.model2)}
\end{Highlighting}
\end{Shaded}

\begin{verbatim}
## 
## Call:
## glm.nb(formula = satell ~ weight, data = crab, init.theta = 0.9310592338, 
##     link = log)
## 
## Deviance Residuals: 
##     Min       1Q   Median       3Q      Max  
## -1.8394  -1.4122  -0.3247   0.4744   2.1279  
## 
## Coefficients:
##             Estimate Std. Error z value Pr(>|z|)    
## (Intercept)  -0.8647     0.4048  -2.136   0.0327 *  
## weight        0.7603     0.1578   4.817 1.45e-06 ***
## ---
## Signif. codes:  0 '***' 0.001 '**' 0.01 '*' 0.05 '.' 0.1 ' ' 1
## 
## (Dispersion parameter for Negative Binomial(0.9311) family taken to be 1)
## 
##     Null deviance: 216.43  on 172  degrees of freedom
## Residual deviance: 196.16  on 171  degrees of freedom
## AIC: 754.64
## 
## Number of Fisher Scoring iterations: 1
## 
## 
##               Theta:  0.931 
##           Std. Err.:  0.168 
## 
##  2 x log-likelihood:  -748.644
\end{verbatim}

The equation for the neg binomial model is log(mu2) = -0.8647 + 0.7603x.
Here, the dispersion parameter is 0.931 with a standard error of 0.168.
Also, the ratio of the residual deviance to df is \textasciitilde{} 1.15
(overdispersion but not drastic). These values suggest that the neg
binomial model is a better fit for these data.

\begin{enumerate}
\def\labelenumi{\alph{enumi})}
\setcounter{enumi}{1}
\item
\end{enumerate}

\begin{Shaded}
\begin{Highlighting}[]
\KeywordTok{confint}\NormalTok{(crab.model2)}
\end{Highlighting}
\end{Shaded}

\begin{verbatim}
## Waiting for profiling to be done...
\end{verbatim}

\begin{verbatim}
##                  2.5 %      97.5 %
## (Intercept) -1.7542558 0.007184686
## weight       0.4207032 1.113666186
\end{verbatim}

The confidence interval is (0.4207032, 1.113666186).The interval range
is wider for the neg binomial model due to the existance of the
parameter that deals with overdispersion in model.

\hypertarget{section-5}{%
\subsection{3.18}\label{section-5}}

\begin{enumerate}
\def\labelenumi{\alph{enumi})}
\item
  We first consider dimensions. The expected value of Y in this case
  should be the number of arrests. \mu is the rate of arrests {[}number
  of arrests / 1000 fans{]} and t in the number of fans {[}in
  thousands{]} this means that if \(E[Y]=\mu t\) we get the number of
  arrests expected at a game. If we solve for \mu then we get
  \(\mu=E[Y]/t\) which gives us the rate of arrests. This number should
  be an integer greater than 0. \(log(E[Y]/t)=log(\mu)\) would help us
  to enforce the positive parameter. Let us denote \(log(\mu)=\alpha\).
  In R, this is expressed as:
  glm(Arrests\textasciitilde1+offset(log(Attendance)),data=soccer,family=poisson(link=``log''))
\item
  The model for the problem is set up below.
\end{enumerate}

\begin{Shaded}
\begin{Highlighting}[]
\NormalTok{soccer =}\StringTok{ }\KeywordTok{read.csv}\NormalTok{(}\StringTok{"C:}\CharTok{\textbackslash{}\textbackslash{}}\StringTok{Users}\CharTok{\textbackslash{}\textbackslash{}}\StringTok{chadm}\CharTok{\textbackslash{}\textbackslash{}}\StringTok{OneDrive}\CharTok{\textbackslash{}\textbackslash{}}\StringTok{Desktop}\CharTok{\textbackslash{}\textbackslash{}}\StringTok{Fall_2020}\CharTok{\textbackslash{}\textbackslash{}}\StringTok{STATS 578}\CharTok{\textbackslash{}\textbackslash{}}\StringTok{HW}\CharTok{\textbackslash{}\textbackslash{}}\StringTok{soccer.csv"}\NormalTok{,}\DataTypeTok{header=}\OtherTok{TRUE}\NormalTok{)}
\KeywordTok{View}\NormalTok{(soccer)}
\NormalTok{soccer.model=}\KeywordTok{glm}\NormalTok{(Arrests}\OperatorTok{~}\DecValTok{1}\OperatorTok{+}\KeywordTok{offset}\NormalTok{(}\KeywordTok{log}\NormalTok{(Attendance)),}\DataTypeTok{data=}\NormalTok{soccer,}\DataTypeTok{family=}\KeywordTok{poisson}\NormalTok{(}\DataTypeTok{link=}\StringTok{"log"}\NormalTok{))}
\KeywordTok{summary}\NormalTok{(soccer.model)}
\end{Highlighting}
\end{Shaded}

\begin{verbatim}
## 
## Call:
## glm(formula = Arrests ~ 1 + offset(log(Attendance)), family = poisson(link = "log"), 
##     data = soccer)
## 
## Deviance Residuals: 
##     Min       1Q   Median       3Q      Max  
## -12.789   -3.426   -0.938    3.079   10.137  
## 
## Coefficients:
##             Estimate Std. Error z value Pr(>|z|)    
## (Intercept) -0.91028    0.02164  -42.07   <2e-16 ***
## ---
## Signif. codes:  0 '***' 0.001 '**' 0.01 '*' 0.05 '.' 0.1 ' ' 1
## 
## (Dispersion parameter for poisson family taken to be 1)
## 
##     Null deviance: 669.45  on 22  degrees of freedom
## Residual deviance: 669.45  on 22  degrees of freedom
## AIC: 812.62
## 
## Number of Fisher Scoring iterations: 5
\end{verbatim}

Our value of \alpha is -0.91028 which leads to a
\(\hat \mu = e^\alpha =e^{-0.91028}=0.40241\). This rate is the
estimated arrests per thousands of fans attending.

\begin{enumerate}
\def\labelenumi{\alph{enumi})}
\setcounter{enumi}{2}
\tightlist
\item
  The graphs are produced according to the code below.
\end{enumerate}

\begin{Shaded}
\begin{Highlighting}[]
\NormalTok{attend.sort=}\KeywordTok{order}\NormalTok{(soccer}\OperatorTok{$}\NormalTok{Attendance)}
\KeywordTok{plot}\NormalTok{(soccer}\OperatorTok{$}\NormalTok{Arrests}\OperatorTok{~}\NormalTok{soccer}\OperatorTok{$}\NormalTok{Attendance,}\DataTypeTok{pch=}\DecValTok{3}\NormalTok{,}\DataTypeTok{col=}\StringTok{"red"}\NormalTok{)}
\KeywordTok{points}\NormalTok{(soccer.model}\OperatorTok{$}\NormalTok{fitted.values[attend.sort]}\OperatorTok{~}\NormalTok{soccer}\OperatorTok{$}\NormalTok{Attendance[attend.sort],}\DataTypeTok{pch=}\DecValTok{2}\NormalTok{,}\DataTypeTok{col=}\StringTok{"black"}\NormalTok{)}
\end{Highlighting}
\end{Shaded}

\includegraphics{HW3_Melton_And_Brink_files/figure-latex/unnamed-chunk-16-1.pdf}

Using residuals, Aston Villa and Bournemouth are both teams that had an
unexpectedly high arrest rate. Manchester City, Plymouth, Reading, and
Hull City all have lower than expected arrest rates.

\begin{enumerate}
\def\labelenumi{\alph{enumi})}
\setcounter{enumi}{3}
\tightlist
\item
  We first set up the model as shown in the code below.
\end{enumerate}

\begin{Shaded}
\begin{Highlighting}[]
\NormalTok{soccer.model2=}\KeywordTok{glm.nb}\NormalTok{(Arrests}\OperatorTok{~}\DecValTok{1}\OperatorTok{+}\KeywordTok{offset}\NormalTok{(}\KeywordTok{log}\NormalTok{(Attendance)),}\DataTypeTok{data=}\NormalTok{soccer)}
\KeywordTok{summary}\NormalTok{(soccer.model2)}
\end{Highlighting}
\end{Shaded}

\begin{verbatim}
## 
## Call:
## glm.nb(formula = Arrests ~ 1 + offset(log(Attendance)), data = soccer, 
##     init.theta = 3.135631071, link = log)
## 
## Deviance Residuals: 
##     Min       1Q   Median       3Q      Max  
## -2.2049  -0.7464  -0.1857   0.6129   1.5568  
## 
## Coefficients:
##             Estimate Std. Error z value Pr(>|z|)    
## (Intercept)  -0.9052     0.1200  -7.546 4.49e-14 ***
## ---
## Signif. codes:  0 '***' 0.001 '**' 0.01 '*' 0.05 '.' 0.1 ' ' 1
## 
## (Dispersion parameter for Negative Binomial(3.1356) family taken to be 1)
## 
##     Null deviance: 24.15  on 22  degrees of freedom
## Residual deviance: 24.15  on 22  degrees of freedom
## AIC: 244.24
## 
## Number of Fisher Scoring iterations: 1
## 
## 
##               Theta:  3.136 
##           Std. Err.:  0.920 
## 
##  2 x log-likelihood:  -240.236
\end{verbatim}

This model has an \alpha of -0.9052 with a standard error of 0.1200.
This is very similar to the previous poisson model where \alpha was
-0.91028 witha standard error of 0.02164. The standard error in the
negative binomial model has a larger deviation than that of the poisson
model. Therefore, the Poisson does not support sufficient variability.

\hypertarget{section-6}{%
\subsection{3.20}\label{section-6}}

\begin{enumerate}
\def\labelenumi{\alph{enumi})}
\item
\end{enumerate}

We use the code below to format the data and calculate the sample
coronary death rates as requested. The plots of the outputs are shown
below.

\begin{Shaded}
\begin{Highlighting}[]
\NormalTok{Age =}\StringTok{ }\KeywordTok{c}\NormalTok{(}\DecValTok{40}\NormalTok{,}\DecValTok{40}\NormalTok{,}\DecValTok{50}\NormalTok{,}\DecValTok{50}\NormalTok{,}\DecValTok{60}\NormalTok{,}\DecValTok{60}\NormalTok{,}\DecValTok{70}\NormalTok{,}\DecValTok{70}\NormalTok{,}\DecValTok{80}\NormalTok{,}\DecValTok{80}\NormalTok{)}
\NormalTok{Smoke =}\StringTok{ }\KeywordTok{rep}\NormalTok{(}\KeywordTok{c}\NormalTok{(}\StringTok{"Y"}\NormalTok{,}\StringTok{"N"}\NormalTok{),}\DecValTok{5}\NormalTok{)}
\NormalTok{PersonYrs =}\StringTok{ }\KeywordTok{c}\NormalTok{(}\DecValTok{52407}\NormalTok{,}\DecValTok{18793}\NormalTok{,}\DecValTok{43248}\NormalTok{,}\DecValTok{10673}\NormalTok{,}\DecValTok{28612}\NormalTok{,}\DecValTok{5710}\NormalTok{,}\DecValTok{12663}\NormalTok{,}\DecValTok{2585}\NormalTok{,}\DecValTok{5317}\NormalTok{,}\DecValTok{1462}\NormalTok{)}
\NormalTok{Deaths =}\StringTok{ }\KeywordTok{c}\NormalTok{(}\DecValTok{32}\NormalTok{,}\DecValTok{2}\NormalTok{,}\DecValTok{104}\NormalTok{,}\DecValTok{12}\NormalTok{,}\DecValTok{206}\NormalTok{,}\DecValTok{28}\NormalTok{,}\DecValTok{186}\NormalTok{,}\DecValTok{28}\NormalTok{,}\DecValTok{102}\NormalTok{,}\DecValTok{31}\NormalTok{)}

\NormalTok{smoking =}\StringTok{ }\KeywordTok{data.frame}\NormalTok{(Age,Smoke,PersonYrs,Deaths)}
\NormalTok{smoking}\OperatorTok{$}\NormalTok{Smoker=}\DecValTok{0}
\NormalTok{smoking}\OperatorTok{$}\NormalTok{Smoker[Smoke}\OperatorTok{==}\StringTok{"Y"}\NormalTok{] =}\StringTok{ }\DecValTok{1}
\NormalTok{smoking}\OperatorTok{$}\NormalTok{logPersonYrs =}\StringTok{ }\KeywordTok{log}\NormalTok{(smoking}\OperatorTok{$}\NormalTok{PersonYrs}\OperatorTok{/}\DecValTok{1000}\NormalTok{);}
\NormalTok{smoking}\OperatorTok{$}\NormalTok{Rate =}\StringTok{ }\NormalTok{smoking}\OperatorTok{$}\NormalTok{Deaths}\OperatorTok{/}\NormalTok{(smoking}\OperatorTok{$}\NormalTok{PersonYrs}\OperatorTok{/}\DecValTok{1000}\NormalTok{);}
\NormalTok{smoking}\OperatorTok{$}\NormalTok{Age2 =smoking}\OperatorTok{$}\NormalTok{Age  }\CommentTok{#use this for part c?;}

\KeywordTok{library}\NormalTok{(ggplot2)}
\KeywordTok{qplot}\NormalTok{(Age,Rate, }\DataTypeTok{data =}\NormalTok{ smoking, }\DataTypeTok{shape =}\NormalTok{ Smoke)}
\end{Highlighting}
\end{Shaded}

\includegraphics{HW3_Melton_And_Brink_files/figure-latex/unnamed-chunk-18-1.pdf}

\begin{Shaded}
\begin{Highlighting}[]
\CommentTok{#Ratio Plot}

\NormalTok{nSmk =}\StringTok{ }\NormalTok{smoking[smoking}\OperatorTok{$}\NormalTok{Smoke}\OperatorTok{==}\StringTok{"N"}\NormalTok{,]}
\NormalTok{Smk =}\StringTok{ }\NormalTok{smoking[smoking}\OperatorTok{$}\NormalTok{Smoke}\OperatorTok{==}\StringTok{"Y"}\NormalTok{,]}
\NormalTok{Smk}\OperatorTok{$}\NormalTok{Ns.rate =}\StringTok{ }\NormalTok{nSmk}\OperatorTok{$}\NormalTok{Rate}
\NormalTok{Smk}\OperatorTok{$}\NormalTok{Ratio =}\StringTok{ }\NormalTok{Smk}\OperatorTok{$}\NormalTok{Rate}\OperatorTok{/}\NormalTok{Smk}\OperatorTok{$}\NormalTok{Ns.rate  }\CommentTok{#smoking coronary death rate/non-smoking death rate.}

\KeywordTok{qplot}\NormalTok{(Age,Ratio, }\DataTypeTok{data =}\NormalTok{ Smk)}
\end{Highlighting}
\end{Shaded}

\includegraphics{HW3_Melton_And_Brink_files/figure-latex/unnamed-chunk-18-2.pdf}

We can see in these figures that the ratio of heart attacks between
smokers and nonsmokers is highly nonlinear. The ratio of younger smokers
to younger non-smokers who have heart attacks is much higher than the
ratio of older smokers to older non-smokers. This could be because as
shown in the first plot, the rate of heart attack is much higher for
people in their 80's. This is clearly a variable relationship.

\begin{enumerate}
\def\labelenumi{\alph{enumi})}
\setcounter{enumi}{1}
\item
\end{enumerate}

We want a poisson model that has a total of six parameters. The equation
that we are asked for comes in the form:

loga(μ/t)= β\_0+ β\_S I(Smoker=Yes)+β\_(Age=40) I(Age=40)+β\_(Age=50)
I(Age=50)+β\_(Age=60) I(Age=60)+β\_(Age=70) I(Age=70)

This model is set up by the code shown below.

\begin{Shaded}
\begin{Highlighting}[]
\KeywordTok{is.factor}\NormalTok{(smoking}\OperatorTok{$}\NormalTok{Smoke)}
\end{Highlighting}
\end{Shaded}

\begin{verbatim}
## [1] FALSE
\end{verbatim}

\begin{Shaded}
\begin{Highlighting}[]
\KeywordTok{is.factor}\NormalTok{(smoking}\OperatorTok{$}\NormalTok{Age)}
\end{Highlighting}
\end{Shaded}

\begin{verbatim}
## [1] FALSE
\end{verbatim}

\begin{Shaded}
\begin{Highlighting}[]
\NormalTok{partb =}\StringTok{ }\KeywordTok{glm}\NormalTok{(Deaths }\OperatorTok{~}\StringTok{ }\KeywordTok{factor}\NormalTok{(Age)}\OperatorTok{+}\NormalTok{Smoke}\OperatorTok{+}\KeywordTok{offset}\NormalTok{(logPersonYrs),}\DataTypeTok{data=}\NormalTok{smoking,}\DataTypeTok{family=}\KeywordTok{poisson}\NormalTok{(}\DataTypeTok{link=}\StringTok{"log"}\NormalTok{))}
\KeywordTok{summary}\NormalTok{(partb)}
\end{Highlighting}
\end{Shaded}

\begin{verbatim}
## 
## Call:
## glm(formula = Deaths ~ factor(Age) + Smoke + offset(logPersonYrs), 
##     family = poisson(link = "log"), data = smoking)
## 
## Deviance Residuals: 
##        1         2         3         4         5         6         7         8  
##  0.90176  -2.18005   0.51036  -1.30797   0.05133  -0.13786  -0.08734   0.22886  
##        9        10  
## -0.91239   1.91906  
## 
## Coefficients:
##               Estimate Std. Error z value Pr(>|z|)    
## (Intercept)    -1.0116     0.1918  -5.275 1.32e-07 ***
## factor(Age)50   1.4840     0.1951   7.606 2.82e-14 ***
## factor(Age)60   2.6275     0.1837  14.301  < 2e-16 ***
## factor(Age)70   3.3505     0.1848  18.131  < 2e-16 ***
## factor(Age)80   3.7001     0.1922  19.250  < 2e-16 ***
## SmokeY          0.3545     0.1074   3.302  0.00096 ***
## ---
## Signif. codes:  0 '***' 0.001 '**' 0.01 '*' 0.05 '.' 0.1 ' ' 1
## 
## (Dispersion parameter for poisson family taken to be 1)
## 
##     Null deviance: 935.091  on 9  degrees of freedom
## Residual deviance:  12.134  on 4  degrees of freedom
## AIC: 79.202
## 
## Number of Fisher Scoring iterations: 4
\end{verbatim}

\begin{Shaded}
\begin{Highlighting}[]
\KeywordTok{deviance}\NormalTok{(partb)}
\end{Highlighting}
\end{Shaded}

\begin{verbatim}
## [1] 12.13391
\end{verbatim}

From the equation, we can see that there is only one term of smoking vs
not and the rest of the terms are age dependant. This means that there
is only one place our model considers smoking or not, and age is not a
factor in smoking or risk from smoking. Considering the fact that we
determined that the risk of heart attack from smoking was not linear in
regards to age, this model would not be appropriate. This model would be
appropriate if the risk of smoking was constant and compounded by the
risk of heart attack correlated with age. The mathmatical way of showing
this model to be insufficent is to consider loga((μ\_s/t)/(μ\_NS/t).
This shows the averae risk of smokers to nonsmokers. In our selected
model this becomes: loga((μ\_s/t)/(μ\_NS/t))=loga(μ\_s/t)-loga(μ\_NS/t)=
(β\_0+ β\_S+β\_(Age=x) ) -(β\_0+β\_(Age=x) )=β\_S

which means the risk is only a function of smoking and not a function of
age at all.

\begin{enumerate}
\def\labelenumi{\alph{enumi})}
\setcounter{enumi}{2}
\item
\end{enumerate}

We can see clearly from the plots in the first part of the problem that
the relationship between smoking and age is not linear and that heart
attack rates are dependent on both factors. Instead of a model with an
additive relationship between age and smoking, it may be more beneficial
to try a multiplicative relationship so that the ratio can change
according to age. This is shown by the equation:

loga(μ/t)= β\_0+ β\_1 * I(Smoker=Yes)+β\_2 Age +β\_3 Age * I(Smoker=Yes)

When we take the same log ratio, loga((μ\_s/t)/(μ\_NS/t), we now get the
equation:

loga((μ\_s/t)/(μ\_NS/t))=loga(μ\_s/t)-loga(μ\_NS/t)= (β\_0+ β\_S+β\_Age
Age+β\_(A * S) Age)-(β\_0+β\_Age Age)=β\_S+β\_(A*S) Age

This makes much more sense as the final risk is a factor of both age and
smoking habit. It also is simplified as there is now only one factor
concerning age. The below code sets up this model and prints the summary
showing the parameters.

\begin{Shaded}
\begin{Highlighting}[]
\NormalTok{partc =}\StringTok{ }\KeywordTok{glm}\NormalTok{(Deaths }\OperatorTok{~}\StringTok{ }\NormalTok{Age}\OperatorTok{+}\NormalTok{Smoke}\OperatorTok{+}\NormalTok{Age}\OperatorTok{*}\NormalTok{Smoke}\OperatorTok{+}\KeywordTok{offset}\NormalTok{(logPersonYrs),}\DataTypeTok{data=}\NormalTok{smoking,}\DataTypeTok{family=}\KeywordTok{poisson}\NormalTok{())}
\KeywordTok{summary}\NormalTok{(partc)}
\end{Highlighting}
\end{Shaded}

\begin{verbatim}
## 
## Call:
## glm(formula = Deaths ~ Age + Smoke + Age * Smoke + offset(logPersonYrs), 
##     family = poisson(), data = smoking)
## 
## Deviance Residuals: 
##     Min       1Q   Median       3Q      Max  
## -3.8784  -2.1219  -0.2482   1.7184   3.5269  
## 
## Coefficients:
##              Estimate Std. Error z value Pr(>|z|)    
## (Intercept) -5.099946   0.530729  -9.609  < 2e-16 ***
## Age          0.104685   0.007743  13.520  < 2e-16 ***
## SmokeY       2.030674   0.568577   3.572 0.000355 ***
## Age:SmokeY  -0.024899   0.008359  -2.979 0.002894 ** 
## ---
## Signif. codes:  0 '***' 0.001 '**' 0.01 '*' 0.05 '.' 0.1 ' ' 1
## 
## (Dispersion parameter for poisson family taken to be 1)
## 
##     Null deviance: 935.091  on 9  degrees of freedom
## Residual deviance:  59.895  on 6  degrees of freedom
## AIC: 122.96
## 
## Number of Fisher Scoring iterations: 4
\end{verbatim}

\begin{Shaded}
\begin{Highlighting}[]
\KeywordTok{deviance}\NormalTok{(partc)}
\end{Highlighting}
\end{Shaded}

\begin{verbatim}
## [1] 59.89526
\end{verbatim}

Looking at this model, and comparing it to the model in b, we note
several important things. The first is the deviance. In terms of
deviance, the model in b performs better. This is in part because we did
not treate age as a categorical value in the model above but instead
treated it as a continous variable. Logically, the model presented here
makes sense as the risk of heart attack goes up with age and initially
increases with smoking but then the smokng affect lessens as the age
range increases. Since our problem is one of data structure, we could
create a hybrid model with the original model but add the single
quantitaitive model for age and smoking. This is set up in R code below.

\begin{Shaded}
\begin{Highlighting}[]
\NormalTok{partd =}\StringTok{ }\KeywordTok{glm}\NormalTok{(Deaths }\OperatorTok{~}\StringTok{ }\KeywordTok{factor}\NormalTok{(Age)}\OperatorTok{+}\NormalTok{Smoke}\OperatorTok{+}\NormalTok{Age}\OperatorTok{*}\NormalTok{Smoke}\OperatorTok{+}\KeywordTok{offset}\NormalTok{(logPersonYrs),}\DataTypeTok{data=}\NormalTok{smoking,}\DataTypeTok{family=}\KeywordTok{poisson}\NormalTok{())}
\KeywordTok{summary}\NormalTok{(partd)}
\end{Highlighting}
\end{Shaded}

\begin{verbatim}
## 
## Call:
## glm(formula = Deaths ~ factor(Age) + Smoke + Age * Smoke + offset(logPersonYrs), 
##     family = poisson(), data = smoking)
## 
## Deviance Residuals: 
##        1         2         3         4         5         6         7         8  
##  0.27124  -0.87753  -0.06894   0.20752  -0.21865   0.62006   0.17544  -0.43806  
##        9        10  
## -0.00332   0.00603  
## 
## Coefficients: (1 not defined because of singularities)
##                Estimate Std. Error z value Pr(>|z|)    
## (Intercept)   -1.677891   0.304983  -5.502 3.76e-08 ***
## factor(Age)50  1.734570   0.213251   8.134 4.16e-16 ***
## factor(Age)60  3.148363   0.253346  12.427  < 2e-16 ***
## factor(Age)70  4.142029   0.319253  12.974  < 2e-16 ***
## factor(Age)80  4.730990   0.383962  12.322  < 2e-16 ***
## SmokeY         2.371211   0.657595   3.606 0.000311 ***
## Age                  NA         NA      NA       NA    
## SmokeY:Age    -0.030874   0.009726  -3.174 0.001502 ** 
## ---
## Signif. codes:  0 '***' 0.001 '**' 0.01 '*' 0.05 '.' 0.1 ' ' 1
## 
## (Dispersion parameter for poisson family taken to be 1)
## 
##     Null deviance: 935.0909  on 9  degrees of freedom
## Residual deviance:   1.5464  on 3  degrees of freedom
## AIC: 70.614
## 
## Number of Fisher Scoring iterations: 4
\end{verbatim}

\begin{Shaded}
\begin{Highlighting}[]
\KeywordTok{deviance}\NormalTok{(partd)}
\end{Highlighting}
\end{Shaded}

\begin{verbatim}
## [1] 1.546438
\end{verbatim}

\begin{enumerate}
\def\labelenumi{\alph{enumi})}
\setcounter{enumi}{3}
\item
\end{enumerate}

This section says to compare the models in parts b and c.~Some of this
discussion can be found in part c, so we pick up discussing the hybrid
model that we programmed. In this model, there are 10 coefficients
present as displayed above. This means that the model is directly
modeling all 10 observations. This model will give us good results, but
is most likely overfit.

\hypertarget{problem-2}{%
\subsection{Problem 2}\label{problem-2}}

\begin{Shaded}
\begin{Highlighting}[]
\NormalTok{Pos =}\StringTok{ }\KeywordTok{read.csv}\NormalTok{(}\StringTok{"C:}\CharTok{\textbackslash{}\textbackslash{}}\StringTok{Users}\CharTok{\textbackslash{}\textbackslash{}}\StringTok{chadm}\CharTok{\textbackslash{}\textbackslash{}}\StringTok{OneDrive}\CharTok{\textbackslash{}\textbackslash{}}\StringTok{Desktop}\CharTok{\textbackslash{}\textbackslash{}}\StringTok{Fall_2020}\CharTok{\textbackslash{}\textbackslash{}}\StringTok{STATS 578}\CharTok{\textbackslash{}\textbackslash{}}\StringTok{HW}\CharTok{\textbackslash{}\textbackslash{}}\StringTok{dvisits.csv"}\NormalTok{,}\DataTypeTok{header=}\OtherTok{TRUE}\NormalTok{)}
\end{Highlighting}
\end{Shaded}

\begin{enumerate}
\def\labelenumi{\alph{enumi})}
\item
\end{enumerate}

\begin{Shaded}
\begin{Highlighting}[]
\NormalTok{Pos.model =}\StringTok{ }\KeywordTok{glm}\NormalTok{(doctorco }\OperatorTok{~}\StringTok{ }\NormalTok{sex }\OperatorTok{+}\StringTok{ }\NormalTok{age }\OperatorTok{+}\StringTok{ }\NormalTok{agesq }\OperatorTok{+}\StringTok{ }\NormalTok{income }\OperatorTok{+}\StringTok{ }\NormalTok{levyplus }\OperatorTok{+}\StringTok{ }\NormalTok{freepoor }\OperatorTok{+}\StringTok{ }\NormalTok{freerepa }\OperatorTok{+}\StringTok{ }\NormalTok{illness }\OperatorTok{+}\StringTok{ }\NormalTok{actdays }\OperatorTok{+}\StringTok{ }\NormalTok{hscore }\OperatorTok{+}\StringTok{ }\NormalTok{chcond1 }\OperatorTok{+}\StringTok{ }\NormalTok{chcond2, }\DataTypeTok{family=}\NormalTok{poisson, }\DataTypeTok{data =}\NormalTok{ Pos)}
\NormalTok{Pos.model}
\end{Highlighting}
\end{Shaded}

\begin{verbatim}
## 
## Call:  glm(formula = doctorco ~ sex + age + agesq + income + levyplus + 
##     freepoor + freerepa + illness + actdays + hscore + chcond1 + 
##     chcond2, family = poisson, data = Pos)
## 
## Coefficients:
## (Intercept)          sex          age        agesq       income     levyplus  
##    -2.22385      0.15688      1.05630     -0.84870     -0.20532      0.12319  
##    freepoor     freerepa      illness      actdays       hscore      chcond1  
##    -0.44006      0.07980      0.18695      0.12685      0.03008      0.11409  
##     chcond2  
##     0.14116  
## 
## Degrees of Freedom: 5189 Total (i.e. Null);  5177 Residual
## Null Deviance:       5635 
## Residual Deviance: 4380  AIC: 6737
\end{verbatim}

\begin{Shaded}
\begin{Highlighting}[]
\KeywordTok{summary}\NormalTok{(Pos.model)}
\end{Highlighting}
\end{Shaded}

\begin{verbatim}
## 
## Call:
## glm(formula = doctorco ~ sex + age + agesq + income + levyplus + 
##     freepoor + freerepa + illness + actdays + hscore + chcond1 + 
##     chcond2, family = poisson, data = Pos)
## 
## Deviance Residuals: 
##     Min       1Q   Median       3Q      Max  
## -2.9170  -0.6862  -0.5743  -0.4839   5.7005  
## 
## Coefficients:
##              Estimate Std. Error z value Pr(>|z|)    
## (Intercept) -2.223848   0.189816 -11.716   <2e-16 ***
## sex          0.156882   0.056137   2.795   0.0052 ** 
## age          1.056299   1.000780   1.055   0.2912    
## agesq       -0.848704   1.077784  -0.787   0.4310    
## income      -0.205321   0.088379  -2.323   0.0202 *  
## levyplus     0.123185   0.071640   1.720   0.0855 .  
## freepoor    -0.440061   0.179811  -2.447   0.0144 *  
## freerepa     0.079798   0.092060   0.867   0.3860    
## illness      0.186948   0.018281  10.227   <2e-16 ***
## actdays      0.126846   0.005034  25.198   <2e-16 ***
## hscore       0.030081   0.010099   2.979   0.0029 ** 
## chcond1      0.114085   0.066640   1.712   0.0869 .  
## chcond2      0.141158   0.083145   1.698   0.0896 .  
## ---
## Signif. codes:  0 '***' 0.001 '**' 0.01 '*' 0.05 '.' 0.1 ' ' 1
## 
## (Dispersion parameter for poisson family taken to be 1)
## 
##     Null deviance: 5634.8  on 5189  degrees of freedom
## Residual deviance: 4379.5  on 5177  degrees of freedom
## AIC: 6737.1
## 
## Number of Fisher Scoring iterations: 6
\end{verbatim}

\begin{enumerate}
\def\labelenumi{\alph{enumi})}
\setcounter{enumi}{1}
\item
\end{enumerate}

\includegraphics{HW3_Melton_And_Brink_files/figure-latex/unnamed-chunk-24-1.pdf}
\includegraphics{HW3_Melton_And_Brink_files/figure-latex/unnamed-chunk-24-2.pdf}
\includegraphics{HW3_Melton_And_Brink_files/figure-latex/unnamed-chunk-24-3.pdf}
\includegraphics{HW3_Melton_And_Brink_files/figure-latex/unnamed-chunk-24-4.pdf}

The line patterns in these data are due to the fact that the data are
discrete.

\begin{enumerate}
\def\labelenumi{\alph{enumi})}
\setcounter{enumi}{2}
\item
\end{enumerate}

\begin{Shaded}
\begin{Highlighting}[]
\KeywordTok{step}\NormalTok{(Pos.model, }\DataTypeTok{direction=}\StringTok{"backward"}\NormalTok{)}
\end{Highlighting}
\end{Shaded}

\begin{verbatim}
## Start:  AIC=6737.08
## doctorco ~ sex + age + agesq + income + levyplus + freepoor + 
##     freerepa + illness + actdays + hscore + chcond1 + chcond2
## 
##            Df Deviance    AIC
## - agesq     1   4380.1 6735.7
## - freerepa  1   4380.3 6735.8
## - age       1   4380.6 6736.2
## <none>          4379.5 6737.1
## - chcond2   1   4382.4 6738.0
## - chcond1   1   4382.5 6738.0
## - levyplus  1   4382.5 6738.1
## - income    1   4385.0 6740.5
## - freepoor  1   4386.2 6741.8
## - sex       1   4387.4 6743.0
## - hscore    1   4388.1 6743.7
## - illness   1   4481.8 6837.4
## - actdays   1   4917.1 7272.7
## 
## Step:  AIC=6735.7
## doctorco ~ sex + age + income + levyplus + freepoor + freerepa + 
##     illness + actdays + hscore + chcond1 + chcond2
## 
##            Df Deviance    AIC
## - freerepa  1   4381.0 6734.5
## <none>          4380.1 6735.7
## - age       1   4383.0 6736.5
## - chcond1   1   4383.2 6736.8
## - levyplus  1   4383.3 6736.9
## - chcond2   1   4383.5 6737.0
## - income    1   4385.0 6738.6
## - freepoor  1   4386.8 6740.4
## - sex       1   4388.0 6741.5
## - hscore    1   4389.1 6742.7
## - illness   1   4481.9 6835.4
## - actdays   1   4917.1 7270.7
## 
## Step:  AIC=6734.53
## doctorco ~ sex + age + income + levyplus + freepoor + illness + 
##     actdays + hscore + chcond1 + chcond2
## 
##            Df Deviance    AIC
## <none>          4381.0 6734.5
## - levyplus  1   4383.4 6735.0
## - chcond1   1   4384.3 6735.9
## - chcond2   1   4384.7 6736.3
## - income    1   4386.7 6738.2
## - age       1   4387.1 6738.7
## - freepoor  1   4389.1 6740.6
## - sex       1   4389.5 6741.0
## - hscore    1   4390.2 6741.8
## - illness   1   4482.7 6834.2
## - actdays   1   4917.6 7269.2
\end{verbatim}

\begin{verbatim}
## 
## Call:  glm(formula = doctorco ~ sex + age + income + levyplus + freepoor + 
##     illness + actdays + hscore + chcond1 + chcond2, family = poisson, 
##     data = Pos)
## 
## Coefficients:
## (Intercept)          sex          age       income     levyplus     freepoor  
##    -2.08906      0.16200      0.35513     -0.19981      0.08369     -0.46960  
##     illness      actdays       hscore      chcond1      chcond2  
##     0.18610      0.12661      0.03112      0.12110      0.15889  
## 
## Degrees of Freedom: 5189 Total (i.e. Null);  5179 Residual
## Null Deviance:       5635 
## Residual Deviance: 4381  AIC: 6735
\end{verbatim}

\begin{enumerate}
\def\labelenumi{\alph{enumi})}
\setcounter{enumi}{3}
\item
\end{enumerate}

This model suggests an elderly person would visit more frequently than
other types of people

\#e)

\begin{Shaded}
\begin{Highlighting}[]
\KeywordTok{predict}\NormalTok{(Pos.model, Pos[}\DecValTok{1}\NormalTok{,], }\DataTypeTok{type=}\StringTok{"response"}\NormalTok{)}
\end{Highlighting}
\end{Shaded}

\begin{verbatim}
##         1 
## 0.3127869
\end{verbatim}

\begin{Shaded}
\begin{Highlighting}[]
\ControlFlowTok{for}\NormalTok{(i }\ControlFlowTok{in} \DecValTok{0}\OperatorTok{:}\DecValTok{10}\NormalTok{) \{}
  \KeywordTok{print}\NormalTok{ (}\StringTok{"Doctor visits = "}\NormalTok{)}
  \KeywordTok{print}\NormalTok{(i)}
  \KeywordTok{print}\NormalTok{(}\KeywordTok{dpois}\NormalTok{(i, }\DataTypeTok{lambda =} \FloatTok{0.153}\NormalTok{))}
  
\NormalTok{  \}}
\end{Highlighting}
\end{Shaded}

\begin{verbatim}
## [1] "Doctor visits = "
## [1] 0
## [1] 0.8581297
## [1] "Doctor visits = "
## [1] 1
## [1] 0.1312938
## [1] "Doctor visits = "
## [1] 2
## [1] 0.01004398
## [1] "Doctor visits = "
## [1] 3
## [1] 0.0005122429
## [1] "Doctor visits = "
## [1] 4
## [1] 1.959329e-05
## [1] "Doctor visits = "
## [1] 5
## [1] 5.995548e-07
## [1] "Doctor visits = "
## [1] 6
## [1] 1.528865e-08
## [1] "Doctor visits = "
## [1] 7
## [1] 3.341661e-10
## [1] "Doctor visits = "
## [1] 8
## [1] 6.390927e-12
## [1] "Doctor visits = "
## [1] 9
## [1] 1.086458e-13
## [1] "Doctor visits = "
## [1] 10
## [1] 1.66228e-15
\end{verbatim}

\begin{enumerate}
\def\labelenumi{\alph{enumi})}
\setcounter{enumi}{5}
\item
\end{enumerate}

\begin{Shaded}
\begin{Highlighting}[]
\NormalTok{LinM <-}\StringTok{ }\KeywordTok{lm}\NormalTok{(doctorco }\OperatorTok{~}\StringTok{ }\NormalTok{sex }\OperatorTok{+}\StringTok{ }\NormalTok{age }\OperatorTok{+}\StringTok{ }\NormalTok{agesq }\OperatorTok{+}\StringTok{ }\NormalTok{income }\OperatorTok{+}\StringTok{ }\NormalTok{levyplus }\OperatorTok{+}\StringTok{ }\NormalTok{freepoor }\OperatorTok{+}\StringTok{ }\NormalTok{freerepa }\OperatorTok{+}\StringTok{ }\NormalTok{illness }\OperatorTok{+}\StringTok{ }\NormalTok{actdays }\OperatorTok{+}\StringTok{ }\NormalTok{hscore }\OperatorTok{+}\StringTok{ }\NormalTok{chcond1 }\OperatorTok{+}\StringTok{ }\NormalTok{chcond2, }\DataTypeTok{data=}\NormalTok{Pos)}
\NormalTok{LinM}
\end{Highlighting}
\end{Shaded}

\begin{verbatim}
## 
## Call:
## lm(formula = doctorco ~ sex + age + agesq + income + levyplus + 
##     freepoor + freerepa + illness + actdays + hscore + chcond1 + 
##     chcond2, data = Pos)
## 
## Coefficients:
## (Intercept)          sex          age        agesq       income     levyplus  
##    0.027632     0.033811     0.203201    -0.062103    -0.057323     0.035179  
##    freepoor     freerepa      illness      actdays       hscore      chcond1  
##   -0.103314     0.033241     0.059946     0.103192     0.016976     0.004384  
##     chcond2  
##    0.041617
\end{verbatim}

\begin{Shaded}
\begin{Highlighting}[]
\KeywordTok{summary}\NormalTok{(LinM)}
\end{Highlighting}
\end{Shaded}

\begin{verbatim}
## 
## Call:
## lm(formula = doctorco ~ sex + age + agesq + income + levyplus + 
##     freepoor + freerepa + illness + actdays + hscore + chcond1 + 
##     chcond2, data = Pos)
## 
## Residuals:
##     Min      1Q  Median      3Q     Max 
## -2.1352 -0.2588 -0.1435 -0.0433  7.0327 
## 
## Coefficients:
##              Estimate Std. Error t value Pr(>|t|)    
## (Intercept)  0.027632   0.072220   0.383  0.70202    
## sex          0.033811   0.021604   1.565  0.11764    
## age          0.203201   0.410016   0.496  0.62020    
## agesq       -0.062103   0.458716  -0.135  0.89231    
## income      -0.057323   0.033089  -1.732  0.08326 .  
## levyplus     0.035179   0.024882   1.414  0.15748    
## freepoor    -0.103314   0.052471  -1.969  0.04901 *  
## freerepa     0.033241   0.038157   0.871  0.38371    
## illness      0.059946   0.008357   7.173 8.39e-13 ***
## actdays      0.103192   0.003657  28.216  < 2e-16 ***
## hscore       0.016976   0.005190   3.271  0.00108 ** 
## chcond1      0.004384   0.023740   0.185  0.85349    
## chcond2      0.041617   0.035863   1.160  0.24592    
## ---
## Signif. codes:  0 '***' 0.001 '**' 0.01 '*' 0.05 '.' 0.1 ' ' 1
## 
## Residual standard error: 0.7139 on 5177 degrees of freedom
## Multiple R-squared:  0.2018, Adjusted R-squared:    0.2 
## F-statistic: 109.1 on 12 and 5177 DF,  p-value: < 2.2e-16
\end{verbatim}

\begin{Shaded}
\begin{Highlighting}[]
\KeywordTok{predict}\NormalTok{(LinM, Pos[}\DecValTok{5190}\NormalTok{,])}
\end{Highlighting}
\end{Shaded}

\begin{verbatim}
##      5190 
## 0.1606531
\end{verbatim}

\includegraphics{HW3_Melton_And_Brink_files/figure-latex/unnamed-chunk-28-1.pdf}
\includegraphics{HW3_Melton_And_Brink_files/figure-latex/unnamed-chunk-28-2.pdf}
\includegraphics{HW3_Melton_And_Brink_files/figure-latex/unnamed-chunk-28-3.pdf}
\includegraphics{HW3_Melton_And_Brink_files/figure-latex/unnamed-chunk-28-4.pdf}

Graphically at a distance, these two models seem to be mostly similar in
slope. However, a closer look reveals more nuanced features in both
models.The Gaussian LM slopes slightly negatively until x =
\textasciitilde0.5 and then reverses slope direction, eventually forming
a gradual shape similar to a square root equation. The Poisson GLM
appears to have a gradual bell or Gaussian shaped curve shape.

\end{document}
